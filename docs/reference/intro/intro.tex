\let\clearforchapter\par % cheating, but saves some space

\chapter{Introduction}

Scout is an experimental programming language that combines sequential and parallel
general-purpose constructs with data analysis and visualization-centric features.  
The language is an extension to the C programming language, but 
it has also been influenced by many other languages and is fundamentally a higher-level 
language than C.  In this manual we assume the reader is familiar with parallel 
programming and the basics of C.  In this chapter we give a brief
introduction to Scout's main features.
The following chapters will provide more details about Scout's abstract 
computational data structures and the parallel, data analysis, visualization and plotting 
constructs.

\section{Scout and the C Programming Language}
\label{ch1:scout-clang}

Scout extends the C programming language to support a new set of abstractions and
domain-specific constructs.  Some of the main domain-specific constructs in Scout are
explicit mesh declarations, instantiations and parallel
computation over the various components of the mesh (e.g. cells, vertices, edges) or
over array elements.  Scout builds upon the fundamental types of C to include higher-level 
abstractions -- such 
as computational meshes and the associated fields stored on a mesh.  From this perspective, 
Scout is a higher-level language than C.

Since Scout is based on the open-source Clang and LLVM Compiler Infrastructure, all of 
the C language features are available. 
Scout extends the fundamental data types
available in C to include vector types of two, three, and four components.  Scout's 
syntax for vector types follows closely with those used by the OpenGL Shading Language
and NVIDIA's C for CUDA.  Section~\ref{ch:datatypes} below covers these topics in more 
detail.

\section{Getting Started}
\label{ch1:started}

The best way to learn any programming language is to get your hands dirty.  The classic
{hello world} program does not need to rewritten for Scout, since Scout fully supports
C.  Therefore, as an introductory program to Scout, Listing~\ref{lst-forallmesh}  
illustrates a simple one-dimensional mesh transformation.  It uses the \texttt{mesh}
abstraction, the mesh's associated fields and the parallel \texttt{forall} construct.

\par\bigskip
\begin{lstlisting}[label=lst-forallmesh,caption={Simple mesh example.}]
uniform mesh MyMesh{
  cells:
    float a;
    float b;
    float sum;
};

int main(int argc, char** argv){

  MyMesh m[512];

  // initialization of a and b not shown

  forall cells c in m {
    sum = a + b;
  }

  return 0;
}
\end{lstlisting}
\par\bigskip\noindent

\section{Compiling a Scout Program}
\label{ch1:compiling}

Scout's compiler follows the characteristics of traditional (Unix) command line interfaces.  In 
the following example, the simple mesh program from Listing~\ref{lst-forallmesh} is compiled and 
executed from the command prompt:

\par\bigskip
\fcolorbox{scoutblue}{lightgray}{
  \parbox{\textwidth-10pt}{%
    \texttt{\small \$ scc simplemesh.sc} \cmdcmt{compiler produces default executable ''a.out''...}\\      
    \texttt{\small \$ ./a.out}\\
    \texttt{\small \$ \_ }
  }
}
\par\bigskip\noindent

The previous compilation of the Scout program creates an executable that is targeted to run
sequentially on a CPU.  

%Scout can also be enabled to run CPU multithreading via the \texttt{-mt}
%option to \texttt{scc} or run on an NVIDIA GPU via the \texttt{-gpu} option to \texttt{scc}.

Scout can be enabled to run on an NVIDIA GPU via the \texttt{-gpu} option to \texttt{scc}.

% Do we have an scc man page anywhere?
%See the on-line \texttt{scc} man page, or Appendix~\ref{app-scc} for more information on the
%available command line options.

See the output to \texttt{scc -help} or Appendix~\ref{app-scc} for more information on the
available command line options.
