\let\clearforchapter\par % cheating, but saves some space


\chapter{Visualization Constructs}

A key feature of the Scout language is the incorporation of visualization and
rendering operations directly within the syntax and semantics of the language 
as first-class constructs.  Listing~\ref{renderall1} shows a simple example.

\par\bigskip
\begin{lstlisting}[float=h,label=renderall1,
	caption={A \texttt{renderall} loop construct.}]
	// Example visualization construct for rendering mesh cells. 
	uniform mesh meshType {
		cells: float a, b;
	};

  meshType myMesh[16];

	renderall cells c of myMesh where (a < 0.5)
  {
			color = rgb(1.0, 0.0, 0.0);
		else
			color = rgb(0.0, 0.0, 1.0);
	}
\end{lstlisting}
\par\bigskip\noindent

In the case where the mesh is three-dimensional, the body of the \texttt{renderall}
is essentially the transfer function for volume rendering.

Isosurfacing is a built-in visualization function provided by Scout.
Listing~\ref{renderall2} shows an example of isosurface computation.
Note isosurfacing is not yet implemented.

\par\bigskip
\begin{lstlisting}[float=h,label=renderall2,
	caption={A \texttt{renderall} loop for viewing an isosurface.}]
	
	uniform mesh meshType{
		cells: float a, b;
	};

  meshType myMesh[16];
	
    // Compute two isosurfaces (values 0.5 and 1.0) and store them 
    // in an unstructured mesh.
	unstructured mesh mySurfaceType;
  mySurfaceType mySurface = isosurface(myMesh.a, {0.5, 1.0});

	renderall faces f of mySurface { // "solid" surface
		...
	}
	
	renderall edges e of mySurface {  // wireframe
		...
	}
	
	renderall cells c of myMesh {  // how do we volume render?
		// If the selected mesh is three-dimensional 
		// we just assume this is a volume rendering
		// construct.
		...
	}
\end{lstlisting}
\par\bigskip\noindent

The \texttt{camera} is also a Scout construct.  Cameras are constructed as in
Listing~\ref{camera1}.

\par\bigskip
\begin{lstlisting}[float=h,label=camera1,
	caption={How to construct a \texttt{camera}.}]
  float3 mypos = float3(-300.0f, -300.0f, -300.0f);
  float3 mylookat = float3(0.0f, 0.0f, 0.0f);
  float3 myup = float3(0.0f, 0.0f, -1.0f);

  camera cam {
    near = 70.0;
    far = 500.0;
    fov = 40.0;
    pos = mypos;
    lookat = mylookat;
    up = myup;
  };
\end{lstlisting}
\par\bigskip\noindent

%% TBD
%The \texttt{window} is also a Scout construct.  Windows are constructed so that images can
%be displayed on them given a camera.  Windows can have multiple viewports within them.  
%A \texttt{viewport} has a width and height and offset from lower left corner.
%Windows and viewports are constructed as shown in Listing~\ref{window1}. 

%\par\bigskip
%\begin{lstlisting}[float=h,label=window1,
%	caption={How to declare a \texttt{window}.}]
%  window win[1024,1024] {
%    background  = hsv(0.1, 0.2, 0.3);
%    save_frames = true;
%    filename    = "heat2d-####.png";
%  };
%
%  viewport aviewport[512, 512] {
%    x_position = 10.0;  // offset from lower left corner
%    y_position = 10.0;
%  };
%
%  awindow.add(aviewport);
%
%  awindow.display();
%
%  // assume a camera and a three-dimensional mesh have been defined
%
%  renderall cells c of amesh with acamera onto aviewport {
%  // ....
%  }
%
%  awindow.delete();  // deletes viewports too
%\end{lstlisting} 
%\par\bigskip\noindent
%%

